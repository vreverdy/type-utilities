%% ================== ISO WG21 C++ PROPOSAL: TYPE UTILITIES ================= %%
%% Project:         Type Utilities
%% Name:            p1081r0.tex
%% Description:     On empty structs in the standard library
%% Creator:         Vincent Reverdy
%% Contributors:    Vincent Reverdy [2018]
%% License:         BSD 3-Clause License
%% ========================================================================== %%



%% =============================== PARAMETERS =============================== %%
% DOCUMENT PARAMETERS
\newcommand{\docno}{P1081R0}
\newcommand{\prevdocno}{}%{P1081R0}
\newcommand{\doctitle}{On empty structs in the standard library}
\newcommand{\docauthor}{Vincent Reverdy}
\newcommand{\doccoauthor}{Collin Gress}
\newcommand{\presfile}{p1081r0_presentation.pdf}
\newcommand{\reldate}{\today}
\newcommand{\firstlibchapter}{language.support}
\newcommand{\lastlibchapter}{thread}
%% ========================================================================== %%



%% ============================== CONFIGURATION ============================= %%
% DOCUMENT TYPE 
% \documentclass[letterpaper,oneside,openany]{memoir}
\documentclass[ebook,10pt,oneside,openany,final]{memoir}
% \includeonly{declarations}

% PACKAGES
\usepackage[american]{babel}
\usepackage[iso,american]{isodate}
\usepackage[final]{listings}
\usepackage{longtable}
\usepackage{ltcaption}
\usepackage{relsize}
\usepackage{textcomp}
\usepackage{underscore}
\usepackage{parskip}
\usepackage{array}
\usepackage[normalem]{ulem}
\usepackage{enumitem}
\usepackage{color}
\usepackage{amsmath}
\usepackage{mathrsfs}
\usepackage{microtype}
\usepackage{multicol}
\usepackage{xspace}
\usepackage{lmodern}
\usepackage[T1]{fontenc}
\usepackage[pdftex, final]{graphicx}
\usepackage[final]{pdfpages}
\usepackage[pdftex,
            pdftitle={\docno},
            pdfsubject={\doctitle},
            pdfcreator={\docauthor},
            bookmarks=true,
            bookmarksnumbered=true,
            pdfpagelabels=true,
            pdfpagemode=UseOutlines,
            pdfstartview=FitH,
            linktocpage=true,
            colorlinks=true,
            linkcolor=blue,
            plainpages=false
           ]{hyperref}
\usepackage{memhfixc}
\usepackage[active,
            header=false,
            handles=false,
            copydocumentclass=false,
            generate=std-gram.ext,
            extract-cmdline={gramSec},
            extract-env={bnftab,simplebnf,bnf,bnfkeywordtab}
            ]{extract}

% COMMANDS
\renewcommand\RSsmallest{5.5pt}

% STYLE
\input{layout}
\input{styles}
\input{macros}
\input{tables}

% INDEX
\makeindex[generalindex]
\makeindex[libraryindex]
\makeindex[grammarindex]
\makeindex[impldefindex]
\makeglossary[xrefindex]

% HYPERREF FIXES
\pdfstringdefDisableCommands{\def\smaller#1{#1}}
\pdfstringdefDisableCommands{\def\textbf#1{#1}}
\pdfstringdefDisableCommands{\def\raisebox#1{}}
\pdfstringdefDisableCommands{\def\hspace#1{}}

% HYPHENATION
\hyphenation{tem-plate ex-am-ple in-put-it-er-a-tor name-space non-zero}

% LIGATURES
\DisableLigatures{encoding = T1, family = tt*}

% TABLES
\newenvironment{libreqtab2b}[2]
{
 \begin{LongTable}
 {#1}{#2}
 {x{.34\hsize}x{.60\hsize}}
}
{
 \end{LongTable}
}
%% ========================================================================== %%



%% ================================ DOCUMENT ================================ %%
% DOCUMENT BEGINNING
\begin{document}

\thispagestyle{empty}
\begingroup
\def\hd{%
    \begin{tabular}{ll}
    \textbf{Document number:} & \docno \\
    \textbf{Revises:}         & \prevdocno \\
    \textbf{Date:}            & \reldate \\
    \textbf{Project:}         & ISO JTC1/SC22/WG21: Programming Language C++ \\
    \textbf{Audience:}        & LEWG \\
    \textbf{Reply to:}        & \docauthor~{}and \doccoauthor \\
                              & University of Illinois at Urbana-Champaign \\
                              & vince.rev@gmail.com
    \end{tabular}
}
\newlength{\hdwidth}
\settowidth{\hdwidth}{\hd}
\begin{minipage}{\hdwidth}\hd\end{minipage}
\endgroup

\vspace{2.5cm}
\begin{center}
\textbf{\Huge\doctitle}
\end{center}
\vfill
\textbf{Note: this is an early draft. It's known to be incomplet and
  incorrekt, and it has lots of
  b\kern-1.2pta\kern1ptd\hspace{1.5em}for\kern-3ptmat\kern0.6ptti\raise0.15ex\hbox{n}g.}
\newpage

\chapterstyle{cppstd}
\pagestyle{cpppage}
%% -------------------------------------------------------------------------- %%
% FRONT MATTER
\frontmatter

\chapter*{Abstract}
In this short paper, we discuss the need for empty structs in the standard library as multipurpose helper classes, and the different design options available.

\phantomsection
\setcounter{tocdepth}{2}
\pdfbookmark{\contentsname}{toctargret}
\hypertarget{toctarget}{\tableofcontents*}
%% -------------------------------------------------------------------------- %%
% MAIN MATTER
\mainmatter
\setglobalstyles
%% -------------------------------------------------------------------------- %%
% PROPOSAL
\rSec0[proposal]{Proposal}
%% -------------------------------------------------------------------------- %%
\rSec1[proposal.problem]{The problem}
An empty structure such as:
\begin{codeblock}
struct empty_struct {};
\end{codeblock}
is a general purpose utility that can be used in many situations. Examples include a default type in contexts requiring a class, a blank type for variant, a default tag class, and an empty base class. The \href{https://www.boost.org/doc/libs/1_67_0/boost/blank.hpp}{Boost} libraries provide a structure called \tcode{blank} for this purpose. In C++17, the standard library provides two main empty structures:
\begin{itemize}
\item \tcode{tuple<>}: however, this is not a ``perfectly'' basic empty structure since it has a member function \tcode{swap} and tuple-related functions have been specialized on it
\item \tcode{monostate}: however, it is currently considered to be a variant helper class, and is put in \tcode{<variant>} accordingly
\end{itemize}
As a consequence the standard library currently lacks a universal ``default'' empty structure. It also lacks a way to transform any type, or sequence of types into an empty structure. An example of use is conditional inheritance:
\begin{codeblock}
// Basic empty struct
struct empty_struct {};

// A base class
struct base {/* something */};

// A class derived from the base if T satisfies something
template <class T>
struct derived: conditional_t<
    is_integral_v<T>,
    base,
    empty_struct
> {/* something */};

// A class derived from the base if T... satisfy something
template <class... T>
struct multi_derived: conditional_t<
    is_integral_v<T>,
    base,
    empty_struct
>... {/* something */}; // Will fail

// A template empty struct
template <class... T>
struct empty_struct_template {};

// A class derived from the base if T... satisfy something
template <class... T>
struct multi_derived: conditional_t<
    is_integral_v<T>,
    base,
    empty_struct_template<T>
>... {/* something */}; // Will work
\end{codeblock}

But is a ``universal'' empty structure even desirable, and what are the different possible options in terms of design?
%% -------------------------------------------------------------------------- %%
\rSec1[proposal.solutions]{Design options}

There are a few possible solutions:
\begin{itemize}
\item Do nothing and do not introduce a ``universal'' empty structure: let users keep creating their own.
\item Make \tcode{monostate} the ``universal'' empty structure: in this case it has to be transfered from \tcode{<variant>} to \tcode{<type_traits>} or \tcode{<utility>}. What about the template version?
\item Add a new empty structure for each new use case that is being standardized: for example, for conditional inheritance introduce a new \tcode{empty_base} and \tcode{empty_base_template} (or similar) in \tcode{<type_traits>} or \tcode{<utility>}. In that case, should \tcode{empty_base} be defined as a type alias for \tcode{empty_base_template<>}? 
\item Add a new ``universal'' empty structure, with a different name than \tcode{monostate} in \tcode{<type_traits>} or \tcode{<utility>} such as \tcode{empty_struct} and \tcode{empty_struct_template} (or similar). In that case, should \tcode{empty_struct} be defined as a type alias for \tcode{empty_struct_template<>}?
\end{itemize}

In any case, the names of the template and non-template versions of a ``universal'' empty structure have to be bikeshedded. Examples of names include: \tcode{empty_struct}, \tcode{empty_base}, \tcode{empty_tag}, \tcode{empty}, \tcode{blank}, \tcode{nothing}, \tcode{none}, \tcode{null}, \tcode{nil}, \tcode{vacant}, \tcode{primal}\dots

%% -------------------------------------------------------------------------- %%
\rSec1[proposal.implementation]{Implementation}
Regardless of the chosen design, and once the name is adjusted, implementation is straightforward and will consist of something along the line of:

\begin{codeblock}
template <class...> struct empty_struct_template {};    // The template empty struct
struct empty_struct {};                                 // The non-template empty struct, version 0
using empty_struct = empty_struct_template<>;           // The non-template empty struct, version 1
\end{codeblock}

Additional functionalities like comparison operators or hash specialization could also be added to the design.
%% -------------------------------------------------------------------------- %%
\rSec1[proposal.impact]{Impact on the standard}
This proposal is a pure library extension. It does not require changes to any standard classes or functions. All the extensions belong to the \tcode{<type_traits>} header or to the \tcode{<utility>} header depending on design choices. Also, depending on the preferred option, \tcode{monostate} may have to be changed from the \tcode{<variant>} header to the \tcode{<type_traits>} header or to the \tcode{<utility>} header.
%% -------------------------------------------------------------------------- %%
\rSec1[proposal.acknowledgements]{Acknowledgements}

The authors would like to thank the participants to the related discussion on the \href{https://groups.google.com/a/isocpp.org/forum/#!topic/std-proposals/R04CWOjABIQ}{future-proposals} group. This work has been made possible thanks to the National Science Foundation through the awards CCF-1647432 and SI2-SSE-1642411.
%% -------------------------------------------------------------------------- %%
\rSec1[proposal.references]{References}

\href{https://github.com/vreverdy/type-utilities}{A few additional type manipulation utilities}, Vincent Reverdy, \emph{Github} (March 2018)

\href{http://www.open-std.org/jtc1/sc22/wg21/docs/papers/2018/n4727.pdf}{N4727}, Working Draft, Standard for Programming Language C++, Richard Smith, \emph{ISO/IEC JTC1/SC22/WG21} (February 2018)

\href{https://groups.google.com/a/isocpp.org/forum/#!topic/std-proposals/R04CWOjABIQ}{General purpose utilities for template metaprogramming and type manipulation}, ISO C++ Standard - Future Proposals, \emph{Google Groups} (March 2018)

\href{https://www.boost.org/doc/libs/1_66_0/boost/blank.hpp}{Boost blank}, Eric Friedman, \emph{The Boost C++ Libraries} (2003)
%% -------------------------------------------------------------------------- %%
\newpage
\phantomsection
\setcounter{chapter}{2}
\addcontentsline{toc}{chapter}{\protect\numberline{\thechapter}Presentation}
\includepdf[pages=-,scale=2,fitpaper,templatesize={256mm}{192mm}]{\presfile}
%% -------------------------------------------------------------------------- %%
% BACK MATTER
\backmatter
\end{document}
% DOCUMENT END
%% ========================================================================== %%
