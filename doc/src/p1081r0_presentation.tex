%% ================== ISO WG21 C++ PROPOSAL: TYPE UTILITIES ================= %%
%% Project:         Type Utilities
%% Name:            p1081r0_presentation.tex
%% Description:     On empty structs in the standard library: presentation
%% Creator:         Vincent Reverdy
%% Contributors:    Vincent Reverdy [2018]
%% License:         BSD 3-Clause License
%% ========================================================================== %%



%% =============================== PARAMETERS =============================== %%
% DOCUMENT PARAMETERS
\newcommand{\docno}{P1081R0}
\newcommand{\prevdocno}{}%{P1081R0}
\newcommand{\doctitle}{On empty structs in the standard library}
\newcommand{\docauthor}{Vincent Reverdy}
\newcommand{\doccoauthor}{Collin Gress}
\newcommand{\reldate}{\today}
\newcommand{\firstlibchapter}{language.support}
\newcommand{\lastlibchapter}{thread}
%% ========================================================================== %%



%% ============================== CONFIGURATION ============================= %%
% DOCUMENT TYPE 
\documentclass[8pt]{beamer}

% THEME
\usetheme{Frankfurt}
\useinnertheme{rectangles}
\setbeamertemplate{blocks}[default]

% PACKAGES
%\usepackage{packages/fancylayout}
\usepackage{ragged2e}
\usepackage{multicol}
\usepackage{lmodern}
\usepackage{xcolor}
\usepackage{csquotes}
\usepackage{environ}
\usepackage{stmaryrd}
\usepackage{sistyle}
\usepackage[T1]{fontenc}
\usepackage{pdfpages}
\usepackage{comment}
\usepackage{listings}
\usepackage{makecell}
\usepackage{hyperref}
\usepackage{titling}
\usepackage{tikz}
\usepackage{tikzsymbols}
\usetikzlibrary{calc,positioning,arrows.meta}
\beamertemplatenavigationsymbolsempty

% INFORMATION
\title{\doctitle}
\author{\docauthor \\ \doccoauthor}
\date{}

% LISTINGS
\lstset{
    language=C++,
    basicstyle=\ttfamily,
    aboveskip=0pt,
    belowskip=0pt,
    morekeywords={constexpr,noexcept,decltype},
    keywordstyle=\color{blue},
    commentstyle=\color[RGB]{0,128,0},
    showstringspaces=false
}

% NOTEBLOCK
\newcounter{notecounter}
\newenvironment<>{noteblock}[1]{%
\stepcounter{notecounter}%
\setbeamercolor{block title}{fg=white,bg=orange}%
\setbeamercolor{itemize item}{fg=orange!75!yellow}%
\begin{block}#2{#1}}{\end{block}}
%% ========================================================================== %%



%% ================================ DOCUMENT ================================ %%
\begin{document}
%% -------------------------------------------------------------------------- %%
\section*{Introduction}
\subsection{Introduction}
\frame{\titlepage}
%% -------------------------------------------------------------------------- %%
\section*{Overview}
\subsection{Overview}
\begin{frame}[fragile]{Overview}
\begin{block}{Summary}
\justifying
A ``universal'' default empty structure is useful in multiple contexts. The standard library currently does not have one.
\end{block}
\vspace{-0.5\baselineskip}
\begin{exampleblock}{Examples}
\begin{lstlisting}[basicstyle=\ttfamily\scriptsize]
// Variant
struct monostate {}; // empty struct
struct S {S(int i) : i(i) {} int i;};
variant<monostate, S> variant;

// Conditional inheritance
struct empty_base {}; // empty struct
template <class T> struct derived
: conditional_t<is_class_v<T> && !is_final_v<T>, T, empty_base> {};

// Need for a template version
template <class...> struct empty_base_template {}; // empty struct
template <class... T> struct derived
: conditional_t<is_class_v<T> && !is_final_v<T>, T, empty_base_template<T>>... {};
\end{lstlisting}
\end{exampleblock}
\vspace{-0.5\baselineskip}
\begin{alertblock}{Currently}
\vspace{-0.5\baselineskip}
\begin{itemize}
\item \lstinline{tuple<>}: in \lstinline{<tuple>}, has a \lstinline{swap} member, is not intended to be a ``universal'' empty structure 
\item \lstinline{monostate}: in \lstinline{<variant>}, is considered to be a helper class for \lstinline{variant}
\item \lstinline{blank}: the ``universal'' empty structure of the \href{https://www.boost.org/doc/libs/1_66_0/boost/blank.hpp}{Boost} libraries
\end{itemize}
\vspace{-0.5\baselineskip}
\end{alertblock}
\vspace{-0.5\baselineskip}
\begin{block}{Question}
\justifying
Need for a ``universal'' empty structure in \lstinline{<type_traits>} or \lstinline{<utility>}?
\end{block}
\end{frame}
%% -------------------------------------------------------------------------- %%
\section*{Design}
\subsection{Design}
\begin{frame}[fragile]{Design options}
\begin{block}{Option 1: do nothing}
\justifying
Do nothing and do not introduce a ``universal'' empty structure: let users keep creating their own.
\end{block}
\begin{block}{Option 2: promote monostate}
\justifying
Make \lstinline{monostate} the ``universal'' empty structure and transfer it from \lstinline{<variant>} to \lstinline{<type_traits>} or \lstinline{<utility>}.
\end{block}
\begin{block}{Option 3: one empty structure per use case}
\justifying
Add a new empty structure for each new use case that is being standardized such as a new \lstinline{empty_base} for conditional inheritance.
\end{block}
\begin{block}{Option 4: a universal empty structure on top of monostate}
\justifying
Add a new ``universal'' empty structure, with a different name than \lstinline{monostate} in \lstinline{<type_traits>} or \lstinline{<utility>}.
\end{block}
\end{frame}

\begin{frame}[fragile]{Design of the template version}
\begin{exampleblock}{Motivation}
\begin{lstlisting}[basicstyle=\ttfamily\scriptsize]
// Need for a template version
template <class...> struct empty_base_template {};
template <class... T> struct derived
: conditional_t<is_class_v<T> && !is_final_v<T>, T, empty_base_template<T>>... {};
\end{lstlisting}
\end{exampleblock}
\begin{block}{Design question}
\vspace{-0.5\baselineskip}
\begin{itemize}
\item \lstinline!struct empty_struct {}!: the non-template and the template versions are two independent structures
\item \lstinline!using empty_struct = empty_struct_template<>!: the non-template version is an alias of the template version
\end{itemize}
\vspace{-0.5\baselineskip}
\end{block}
\end{frame}
%% -------------------------------------------------------------------------- %%
\section*{Conclusion}
\subsection{Conclusion}

\begin{frame}[fragile]{Conclusion}
\vspace{-0.5\baselineskip}
\begin{noteblock}{What option?}
\vspace{-0.5\baselineskip}
\begin{itemize}
\item Option 1: do nothing
\item Option 2: promote monostate
\item Option 3: one empty structure per use case
\item Option 4: a universal empty structure on top of monostate
\end{itemize}
\vspace{-0.5\baselineskip}
\end{noteblock}
\vspace{-0.5\baselineskip}
\begin{noteblock}{Non-template and template versions}
\vspace{-0.5\baselineskip}
\begin{itemize}
\item the non-template and the template versions are two independent structures?
\item the non-template version is an alias of the template version?
\end{itemize}
\vspace{-0.5\baselineskip}
\end{noteblock}
\vspace{-0.5\baselineskip}
\begin{block}{Bikeshedding}
\vspace{-1.\baselineskip}
\begin{multicols}{3}
\begin{itemize}
\item \lstinline{empty_struct}
\item \lstinline{empty}
\item \lstinline{blank}
\item \lstinline{nothing}
\item \lstinline{none}
\item \lstinline{null}
\item \lstinline{nil}
\item \lstinline{vacant}
\item \lstinline{primal}
\end{itemize}
\end{multicols}
\vspace{-1.\baselineskip}
\begin{itemize}
\item \dots and what about the template version?
\end{itemize}
\vspace{-0.5\baselineskip}
\end{block}
\vspace{-0.5\baselineskip}
\begin{noteblock}{Additional functionalities}
\vspace{-0.5\baselineskip}
\begin{itemize}
\item Comparison operators?
\item Hash specialization?
\end{itemize}
\vspace{-0.5\baselineskip}
\end{noteblock}
\end{frame}

\begin{frame}
\vfill
\centering
\begin{beamercolorbox}[sep=8pt,center,shadow=true,rounded=true]{title}
    \usebeamerfont{title}Thank you for your attention\par%
\end{beamercolorbox}
\vfill  
\end{frame}
%% -------------------------------------------------------------------------- %%
\end{document}
%% ========================================================================== %%
